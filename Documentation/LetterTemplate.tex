\documentclass{letter}
\usepackage{colortbl}
\usepackage[utf8]{inputenc}
\usepackage[ngerman]{babel}
\usepackage{ifthen}

\begin{document}

\newcommand{\debtAmount}{summe_nachzahlung}% Placeholder for the debt amount. Replace <tenant_debt> with the actual value.
\newcommand{\Abrechnungsjahr}{rechnungs_jahr}
\newcommand{\tenantFirstname}{tenant_vorname}
\newcommand{\tenantLastname}{tenant_nachname}
\newcommand{\paydate}{zahlungs_tag}
\newcommand{\overviewTable}{overview_table}

\begin{letter}{
\textbf{Empfänger:} \\
\tenantFirstname \tenantLastname
}

\address{
Andreas Wiese \\
Am Pfarrgärtchen 13 \\
Heusenstamm, 63150
}

\date{\today}

\opening{\textbf{Betreff: Nebenkostenabrechnung für das Jahr <\Abrechnungsjahr>}}

\vspace{20pt}

Sehr geehrter \tenantFirstname \tenantLastname,

anbei übersende ich Ihnen die Nebenkostenabrechnung für das Jahr <\Abrechnungsjahr>.

\ifthenelse{\debtAmount > 0}{
    \ifthenelse{\debtAmount < 5}{
        Aus dieser ergibt sich eine Nachzahlung in Höhe von \debtAmount Euro, welche wir getrost vergessen können.
    }{
        Aus dieser ergibt sich eine Nachzahlung in Höhe von \debtAmount Euro. Ich bitte um eine Überweisung bis zum \paydate auf folgendes Bankkonto:

        \vspace{10pt}

        Bank: Sparkasse ALK

        Kontonummer: DE45 4625 1630 0000 1048 44

        BLZ: 46251630
    }
}{
    Aus dieser ergibt sich eine Rückzahlung in Höhe von \debtAmount Euro, welche ich in den kommenden Tagen auf Ihr mir bekanntes Konto überweisen werde.
}

\vspace{10pt}

Mit freundlichen Grüssen,

Andreas Wiese

\vspace{20pt}

\end{letter}

\newpage
\overviewTable

\end{document}

